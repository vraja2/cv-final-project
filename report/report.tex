\documentclass{article} % For LaTeX2e
\usepackage{nips15submit_e,times}
\usepackage{hyperref}
\usepackage{url}
\usepackage{graphicx}
\usepackage{float}
\usepackage{caption}
\usepackage{subfigure}
%\documentstyle[nips14submit_09,times,art10]{article} % For LaTeX 2.09


\title{Identifying Signs of Diabetic Retinopathy}


\author{
Vignesh Raja\\
Department of Computer Science\\
University of Illinois\\
Urbana, IL 61801 \\
\texttt{vraja2@illinois.edu} \\
\And
Martin Liu \\
Department of Computer Science \\
University of Illinois\\
Urbana, IL 61801 \\
\texttt{mliu2@illinois.edu} \\
}

% The \author macro works with any number of authors. There are two commands
% used to separate the names and addresses of multiple authors: \And and \AND.
%
% Using \And between authors leaves it to \LaTeX{} to determine where to break
% the lines. Using \AND forces a linebreak at that point. So, if \LaTeX{}
% puts 3 of 4 authors names on the first line, and the last on the second
% line, try using \AND instead of \And before the third author name.

\newcommand{\fix}{\marginpar{FIX}}
\newcommand{\new}{\marginpar{NEW}}

\nipsfinalcopy % Uncomment for camera-ready version

\begin{document}


\maketitle

\begin{abstract}
This paper describes some generalized and domain-dependent approaches to automatically detect the severity of diabetic retionapthy (DR) in input eye images. Our generalized approaches include classification with color histograms and SIFT feature vectors, while our domain-dependent solutions attempted to detect specific characteristics of the disease such as exudates and dot hemorrhages using computer vision techniques. Our performance using these techniques ended up being low due to a high variance in the color spectrum of the dataset of eye images.
\end{abstract}

\section{Introduction}

Diabetic retinopathy affects over 93 million people in the world and is a consequence of diabetes and around 40-45\% of Americans who have diabetes suffer from the disease. This disease is serious because it can eventually lead to blindness. However, if the disease is detected at an early enough stage, impairment may be prevented. \\ \\
The current process for identifying cases of diabetic retinopathy is tedious and slow because a trained clinician needs to manually analyze images of the patient’s retina and oftentimes the process of getting the results to the patient are also slow. Thus, an automated system can make this process far more efficient and will improve the lives of those afflicted. \\ \\
The problem we want to solve is as follows. Given an input image of an eye, we want to categorize it into one of 5 categories: 0 - No DR, 1 - Mild, 2 - Moderate, 3 - Severe, and 4 - Proliferative DR. Below are some eye images of each of the classes.
\begin{figure}[H]
	\centering
	\subfigure[Label 0]{\label{fig:a}\includegraphics[height=1.3in,width=1.6in]{./images/10031_left.jpeg}}
	\subfigure[Label 1]{\label{fig:a}\includegraphics[height=1.3in,width=1.6in]{./images/10727_left.jpeg}}
	\subfigure[Label 2]{\label{fig:a}\includegraphics[height=1.3in,width=1.6in]{./images/15138_left.jpeg}}
\end{figure}  
\begin{figure}[H]
	\centering
	\subfigure[Label 3]{\label{fig:a}\includegraphics[height=1.3in,width=1.6in]{./images/19367_left.jpeg}}
	\subfigure[Label 4]{\label{fig:a}\includegraphics[height=1.3in,width=1.6in]{./images/1084_left.jpeg}}
\end{figure} 
For ideas about some domain-dependent features, we referred to a paper titled ``Screening for Diabetic Retinopathy Using Computer Vision and Physiological Markers'' which was written by Christoper E. Hann and colleagues. In this paper, Hann describes methods of detecting exudates and dot hemorrhages; both of these conditions aid in characterizing the severity of the disease. \\ \\
However, an important point to note regarding Hann's work is that he does not categorize images into classes using exudate and hemorrhage features. Instead, he evaluates his identification methods by computing Positive Predictive Value (PPV) and Negative Predictive Value (NPV) scores where a human manually counts the number of exudate regions and dot hemorrhages and these values are compared with the output of the identification algorithm. The PPV and NPV results were very good for both exudate (PPV: 97\%, NPV: 95\%) and hemorrhage (PPV: 96\%, NPV: 100\%) identification tasks. \\ \\
In our work, we utilize these identification methods to build a multiclass classifier. We also attempt to perform classification using generalized approaches such as color histograms and SIFT feature vectors.

\section{Generalized Approaches}

\subsection{Color Histograms}

\subsection{Histograms using SIFT Feature Vectors}

Scale-invariant feature transform (SIFT) is a computer vision technique to describe local features and keypoints in images. It is often used in feature tracking and object recognition. We felt that SIFT features could potentially be a useful method of describing eye images and that local eye characteristics would be represented as keypoints in the SIFT feature vectors. For this project, we used $128 \times 1$ dimensional feature vectors.\\ \\
Our method of using SIFT features is as follows. Because we have 35,126 high resolution eye images in total and using all of these images for training is intractable on our local machines, we cut down our dataset to 1,500 images. We first use VLFeat's SIFT library to generate SIFT feature vectors for each image. This process took around 10 hours on our local machines due to the very high resolution of the images. Since the number of SIFT vectors per image is also initially very large, we randomly select $n=10,20,30,40,50,60,70,$ and $80$ SIFT vectors per image. Next, with $n$ SIFT vectors per image, we constructed a $128 \times (1,500 \times n)$ SIFT matrix. \\ \\
Our next step is to initialize $K=100$ means from this SIFT matrix. We randomly select $100$ vectors from the SIFT matrix and execute $100$ iterations of the K-means clustering algorithm to converge on a final $128 \times 100$ set of means. \\ \\
Next, for each image, we construct a histogram where each bin is a mean and the frequencies for each mean are the number of times that a SIFT vector for the image is closest to that mean. Thus, every image's histogram has $100$ bins. These histograms are created for both training and testing images.\\ \\
With these histograms we now construct a classifier. We try two different classification methods. Our first method is to use a nearest neighbor search with Euclidean distance. Thus, for a test histogram, we simply find the training histogram that has the shortest Euclidean distance to it and assign it the training image's label. Our second method is to train a multiclass SVM using the training histograms as feature vectors and their respective labels. 

\section{Headings: first level}
\label{headings}

First level headings are lower case (except for first word and proper nouns),
flush left, bold and in point size 12. One line space before the first level
heading and 1/2~line space after the first level heading.

\subsection{Headings: second level}

Second level headings are lower case (except for first word and proper nouns),
flush left, bold and in point size 10. One line space before the second level
heading and 1/2~line space after the second level heading.

\subsubsection{Headings: third level}

Third level headings are lower case (except for first word and proper nouns),
flush left, bold and in point size 10. One line space before the third level
heading and 1/2~line space after the third level heading.

\section{Citations, figures, tables, references}
\label{others}

These instructions apply to everyone, regardless of the formatter being used.

\subsection{Citations within the text}

Citations within the text should be numbered consecutively. The corresponding
number is to appear enclosed in square brackets, such as [1] or [2]-[5]. The
corresponding references are to be listed in the same order at the end of the
paper, in the \textbf{References} section. (Note: the standard
\textsc{Bib\TeX} style \texttt{unsrt} produces this.) As to the format of the
references themselves, any style is acceptable as long as it is used
consistently.

As submission is double blind, refer to your own published work in the 
third person. That is, use ``In the previous work of Jones et al.\ [4]'',
not ``In our previous work [4]''. If you cite your other papers that
are not widely available (e.g.\ a journal paper under review), use
anonymous author names in the citation, e.g.\ an author of the
form ``A.\ Anonymous''. 


\subsection{Footnotes}

Indicate footnotes with a number\footnote{Sample of the first footnote} in the
text. Place the footnotes at the bottom of the page on which they appear.
Precede the footnote with a horizontal rule of 2~inches
(12~picas).\footnote{Sample of the second footnote}

\subsection{Figures}

All artwork must be neat, clean, and legible. Lines should be dark
enough for purposes of reproduction; art work should not be
hand-drawn. The figure number and caption always appear after the
figure. Place one line space before the figure caption, and one line
space after the figure. The figure caption is lower case (except for
first word and proper nouns); figures are numbered consecutively.

Make sure the figure caption does not get separated from the figure.
Leave sufficient space to avoid splitting the figure and figure caption.

You may use color figures. 
However, it is best for the
figure captions and the paper body to make sense if the paper is printed
either in black/white or in color.
\begin{figure}[h]
\begin{center}
%\framebox[4.0in]{$\;$}
\fbox{\rule[-.5cm]{0cm}{4cm} \rule[-.5cm]{4cm}{0cm}}
\end{center}
\caption{Sample figure caption.}
\end{figure}

\subsection{Tables}

All tables must be centered, neat, clean and legible. Do not use hand-drawn
tables. The table number and title always appear before the table. See
Table~\ref{sample-table}.

Place one line space before the table title, one line space after the table
title, and one line space after the table. The table title must be lower case
(except for first word and proper nouns); tables are numbered consecutively.

\begin{table}[t]
\caption{Sample table title}
\label{sample-table}
\begin{center}
\begin{tabular}{ll}
\multicolumn{1}{c}{\bf PART}  &\multicolumn{1}{c}{\bf DESCRIPTION}
\\ \hline \\
Dendrite         &Input terminal \\
Axon             &Output terminal \\
Soma             &Cell body (contains cell nucleus) \\
\end{tabular}
\end{center}
\end{table}

\section{Final instructions}
Do not change any aspects of the formatting parameters in the style files.
In particular, do not modify the width or length of the rectangle the text
should fit into, and do not change font sizes (except perhaps in the
\textbf{References} section; see below). Please note that pages should be
numbered.

\section{Preparing PostScript or PDF files}

Please prepare PostScript or PDF files with paper size ``US Letter'', and
not, for example, ``A4''. The -t
letter option on dvips will produce US Letter files.

Fonts were the main cause of problems in the past years. Your PDF file must
only contain Type 1 or Embedded TrueType fonts. Here are a few instructions
to achieve this.

\begin{itemize}

\item You can check which fonts a PDF files uses.  In Acrobat Reader,
select the menu Files$>$Document Properties$>$Fonts and select Show All Fonts. You can
also use the program \verb+pdffonts+ which comes with \verb+xpdf+ and is
available out-of-the-box on most Linux machines.

\item The IEEE has recommendations for generating PDF files whose fonts
are also acceptable for NIPS. Please see
\url{http://www.emfield.org/icuwb2010/downloads/IEEE-PDF-SpecV32.pdf}

\item LaTeX users:

\begin{itemize}

\item Consider directly generating PDF files using \verb+pdflatex+
(especially if you are a MiKTeX user). 
PDF figures must be substituted for EPS figures, however.

\item Otherwise, please generate your PostScript and PDF files with the following commands:
\begin{verbatim} 
dvips mypaper.dvi -t letter -Ppdf -G0 -o mypaper.ps
ps2pdf mypaper.ps mypaper.pdf
\end{verbatim}

Check that the PDF files only contains Type 1 fonts. 
%For the final version, please send us both the Postscript file and
%the PDF file. 

\item xfig "patterned" shapes are implemented with 
bitmap fonts.  Use "solid" shapes instead. 
\item The \verb+\bbold+ package almost always uses bitmap
fonts.  You can try the equivalent AMS Fonts with command
\begin{verbatim}
\usepackage[psamsfonts]{amssymb}
\end{verbatim}
 or use the following workaround for reals, natural and complex: 
\begin{verbatim}
\newcommand{\RR}{I\!\!R} %real numbers
\newcommand{\Nat}{I\!\!N} %natural numbers 
\newcommand{\CC}{I\!\!\!\!C} %complex numbers
\end{verbatim}

\item Sometimes the problematic fonts are used in figures
included in LaTeX files. The ghostscript program \verb+eps2eps+ is the simplest
way to clean such figures. For black and white figures, slightly better
results can be achieved with program \verb+potrace+.
\end{itemize}
\item MSWord and Windows users (via PDF file):
\begin{itemize}
\item Install the Microsoft Save as PDF Office 2007 Add-in from
\url{http://www.microsoft.com/downloads/details.aspx?displaylang=en\&familyid=4d951911-3e7e-4ae6-b059-a2e79ed87041}
\item Select ``Save or Publish to PDF'' from the Office or File menu
\end{itemize}
\item MSWord and Mac OS X users (via PDF file):
\begin{itemize}
\item From the print menu, click the PDF drop-down box, and select ``Save
as PDF...''
\end{itemize}
\item MSWord and Windows users (via PS file):
\begin{itemize}
\item To create a new printer
on your computer, install the AdobePS printer driver and the Adobe Distiller PPD file from
\url{http://www.adobe.com/support/downloads/detail.jsp?ftpID=204} {\it Note:} You must reboot your PC after installing the
AdobePS driver for it to take effect.
\item To produce the ps file, select ``Print'' from the MS app, choose
the installed AdobePS printer, click on ``Properties'', click on ``Advanced.''
\item Set ``TrueType Font'' to be ``Download as Softfont''
\item Open the ``PostScript Options'' folder
\item Select ``PostScript Output Option'' to be ``Optimize for Portability''
\item Select ``TrueType Font Download Option'' to be ``Outline''
\item Select ``Send PostScript Error Handler'' to be ``No''
\item Click ``OK'' three times, print your file.
\item Now, use Adobe Acrobat Distiller or ps2pdf to create a PDF file from
the PS file. In Acrobat, check the option ``Embed all fonts'' if
applicable.
\end{itemize}

\end{itemize}
If your file contains Type 3 fonts or non embedded TrueType fonts, we will
ask you to fix it. 

\subsection{Margins in LaTeX}
 
Most of the margin problems come from figures positioned by hand using
\verb+\special+ or other commands. We suggest using the command
\verb+\includegraphics+
from the graphicx package. Always specify the figure width as a multiple of
the line width as in the example below using .eps graphics
\begin{verbatim}
   \usepackage[dvips]{graphicx} ... 
   \includegraphics[width=0.8\linewidth]{myfile.eps} 
\end{verbatim}
or % Apr 2009 addition
\begin{verbatim}
   \usepackage[pdftex]{graphicx} ... 
   \includegraphics[width=0.8\linewidth]{myfile.pdf} 
\end{verbatim}
for .pdf graphics. 
See section 4.4 in the graphics bundle documentation (\url{http://www.ctan.org/tex-archive/macros/latex/required/graphics/grfguide.ps}) 
 
A number of width problems arise when LaTeX cannot properly hyphenate a
line. Please give LaTeX hyphenation hints using the \verb+\-+ command.


\subsubsection*{Acknowledgments}

Use unnumbered third level headings for the acknowledgments. All
acknowledgments go at the end of the paper. Do not include 
acknowledgments in the anonymized submission, only in the 
final paper. 

\subsubsection*{References}

References follow the acknowledgments. Use unnumbered third level heading for
the references. Any choice of citation style is acceptable as long as you are
consistent. It is permissible to reduce the font size to `small' (9-point) 
when listing the references. {\bf Remember that this year you can use
a ninth page as long as it contains \emph{only} cited references.}

\small{
[1] Alexander, J.A. \& Mozer, M.C. (1995) Template-based algorithms
for connectionist rule extraction. In G. Tesauro, D. S. Touretzky
and T.K. Leen (eds.), {\it Advances in Neural Information Processing
Systems 7}, pp. 609-616. Cambridge, MA: MIT Press.

[2] Bower, J.M. \& Beeman, D. (1995) {\it The Book of GENESIS: Exploring
Realistic Neural Models with the GEneral NEural SImulation System.}
New York: TELOS/Springer-Verlag.

[3] Hasselmo, M.E., Schnell, E. \& Barkai, E. (1995) Dynamics of learning
and recall at excitatory recurrent synapses and cholinergic modulation
in rat hippocampal region CA3. {\it Journal of Neuroscience}
{\bf 15}(7):5249-5262.
}

\end{document}
